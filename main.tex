\documentclass[hidelinks]{article}
\usepackage{tikz}
\usepackage[a4paper,margin=1in]{geometry}
\usepackage{fix-cm}
\usepackage{wrapfig}
\usepackage{float} 
\usepackage{hyperref}
\usepackage{ulem}
\usepackage[document]{ragged2e}
\renewcommand*\familydefault{\sfdefault} 
\pagestyle{plain}

\begin{document}
\vspace*{-5em}
\begin{center}
    \begin{huge}
        Manual of Cyber-physical Systems Lab at Uppsala University\\
    \end{huge}
    \vspace{15pt}
    \begin{large}
        by NIU Xuezhi $<$\href{mailto:xuezhi.niu@it.uu.se}{xuezhi.niu at it.uu.se}$>$
    \end{large}
\end{center}
\vspace*{-1em}

\setlength{\parskip}{13pt plus2pt}
\section*{Overview and available equipments}

\begin{wrapfigure}{r}{0.35\textwidth}
    \begin{center}
        \includegraphics[width=\linewidth]{Figures/MainPhoto.jpeg}
    \end{center}
\end{wrapfigure}

% 
Welcome to the Cyber-Physical Systems Lab at Uppsala University. This manual serves as a guide to understanding the operations, equipment, and protocols within the lab. The Cyber-Physical Lab is dedicated to the exploration, development, and experimentation of systems that integrate computational and physical elements. Our mission is to provide students and researchers with a conducive environment to explore the interdisciplinary field of cyber-physical systems.

Equipment except robots in this lab (up to \today):
\begin{itemize}
    \item Workstation (HP Z2G9 I7-13900K + A4000 \sout{wireless adapter})
    \item TP-Link AX73 Router (Official manual \underline{\href{https://www.tp-link.com/se/support/download/archer-ax73/v2/}{here}})
    \item TP-Link TL-WR841N (Official manual \underline{\href{https://www.tp-link.com/us/user-guides/tl-wr841n_v14/\#ug-sub-title-1}{here}})
    \item Samsung S34C50 Monitor
    \item Kinnarps 6000 Chairs $\times 2$ (Manual \underline{\href{https://www.kinnarps.com/products/seating/task-chairs/60008000/}{here}})
    \item Several Mice and Keyboard from different brands (Wired $\times 2$ + Wireless $\times 2$)
\end{itemize}

As a cyber-physical systems lab, we should have some robots to play with, and implement your genius ideas. However, the foundation of any successful experimentation, especially in a space as interactive and potentially hazardous as ours, is safety. It is crucial to prioritize safety above all.

\subsection*{Lab Safety}
Before we dive into the exciting world of robotics and cyber-physical systems, we must ensure that every lab member is equipped with the knowledge and respect for safety protocols that govern our operations. This commitment to maintaining a secure environment is not just about adhering to rules but about fostering a culture of mindfulness and responsibility towards oneself and others in the lab. 

\textbf{Under no circumstances should any robotic operation be manually interrupted by hand}. Direct physical interaction with moving parts or operational machinery presents a significant risk of injury and can damage the equipment. If there is a need to halt a robot's operation, the first course of action should always be an attempt to interrupt the process via command through the controlling software. If the software fails to respond or an immediate stop is necessary, the next step is to safely power off the equipment. Only after these measures are taken should new operations be initiated.

\textbf{No unattended robotic operations}. It is imperative that robotic operations are not left unattended. When actuating ideas on the robots, your presence is required at all times. This rule ensures that any unexpected issues can be addressed promptly and reduces the risk of accidents or damage to the lab equipment. Unattended operations increase the likelihood of unforeseen incidents, which can lead to potential harm to both the individual and the workspace.

\textbf{Pre-Operation Inspection}. Before initiating any experiment or operation with robotic systems, perform a thorough pre-operation inspection. This includes checking for any signs of wear and tear, ensuring all parts are secured and in their correct positions, and verifying that the software and hardware communication is functioning correctly. Regular inspections help prevent accidents caused by equipment malfunction or failure.

\textbf{Shutdown Before Leaving}. All robotic systems must be properly shut down before leaving the lab. This rule is crucial to prevent any accidental activation or continuation of operations that could occur in the absence of supervision. A powered-down state ensures that the equipment remains safe and secure until it is next used under direct supervision.

\textbf{Also take care of robots}. When working with robots, consider not only your safety but also the well-being of the robots. Abrupt shutdowns or erratic operational commands can lead to wear and tear or even permanent damage to sensitive components. Always shut down the robots gently and as per the recommended procedures when you are done or if you are leaving the lab, even for a short period. This practice extends the lifespan of the robots and maintains their readiness for future experiments.

\subsection*{Robots}
Available Robots in the lab (up to \today):
\begin{itemize}
    \item myCobot 280 Pi $\times 2$ from Elephant Robotics (Official manual \underline{\href{https://docs.elephantrobotics.com/docs/gitbook-en/2-serialproduct/2.1-280/2.1.2-PI.html}{here}})
    \item myAVG $\times 2$ from Elephant Robotics (Official manual \underline{\href{https://docs.elephantrobotics.com/docs/gitbook-en/2-serialproduct/2.5-myAGV.html}{here}})
    \item Nova 5 from Dobot (Official manual \underline{\href{https://www.dobot-robots.com/products/nova-series/nova5.html}{here}})
    \item AI kit 2023 of myCobot from Elephant Robotics (Official manual \underline{\href{https://docs.elephantrobotics.com/docs/gitbook-en/2-serialproduct/2.9-AIkit2023en/introduce.html}{here}})
    \item Gripper RobotiQ 2F85 from Dobot (Official manual \underline{\href{https://robotiq.com/products/2f85-140-adaptive-robot-gripper?ref=nav_product_new_button}{here}})
\end{itemize}

If you want to connect to the robot (except Nova 5), you could try VNC by typing the IP address assigned by the router to have graphical interaction to the ubuntu installed on the robot. There are two ways to get the assigned IP address. This is what you can achieve, as shown in the figure below.

\begin{figure}[H]
    \centering
    \includegraphics[width=0.75\linewidth]{Figures/work_flow.png}
\end{figure}

\begin{itemize}
    \item Check the IP address from the robot by typing 'ip addr' in the Terminal of robot. See the figure below.
\begin{figure}[H]
    \centering
    \includegraphics[width=0.6\linewidth]{Figures/ip_robot.png}
\end{figure}
    \item Check the IP address from the configuration page of routers. The default address for TP-Link router is '192.168.0.1'. See the figure below.
\begin{figure}[H]
    \centering
    \includegraphics[width=0.75\linewidth]{Figures/ip_router.png}
\end{figure}
\end{itemize}

After obtaining the IP address of the robot, input it into VNC to establish remote access. This allows you to work on a more powerful computer than the Raspberry Pi, where the firmware is executed.

\textbf{Dobot Nova 5}
The procedure for connecting to the large robotic arm differs slightly, as it utilizes industry-standard software developed by Dobot, known as DobotStudio Pro. Before using the robot, you need to connect to the Wi-Fi network that the robot broadcast itself, and then connect to '192.168.5.1' to have full features that software can do. Additionally, there is an alternative address, '192.168.1.6', designated for data exchange. Connecting to this address allows the robot to execute commands, but the visual animation of the arm will be disabled.
\begin{figure}[H]
    \centering
    \includegraphics[width=0.8\linewidth]{Figures/dobot_gui.png}
\end{figure}

\newpage
\section*{Routers}
We have two routers in the lab. 

\begin{minipage}[t]{0.5\linewidth}
    TP-Link AX73 Router is configured to have 
    \begin{itemize}
        \item Wi-Fi network: 'Robot\_All'
        \item Password: CPS104264 
        \item Configuration page: 192.168.0.1
        \item Admin password: CPS104264
    \end{itemize}
\end{minipage}%
\begin{minipage}[t]{0.5\linewidth}
    TP-Link TL-WR841N is is configured to have 
    \begin{itemize}
        \item Wi-Fi network: 'Robot\_Net'
        \item Password: 12345678 
        \item Configuration page: 192.168.0.1
        \item Admin password: 123456
    \end{itemize}
\end{minipage}


\section*{Lab Computer}

\section*{Organizing the GitHub Repository or Project}
When it comes to managing a GitHub repository or project, effective organization is crucial for collaboration, clarity, and efficiency. A well-structured repository not only makes it easier for contributors to understand and contribute to your project but also enhances its overall maintainability. In this tutorial, we'll explore best practices and strategies for organizing your GitHub repository or project effectively.

\subsection*{Define Repository Structure}
Before diving into organizing your repository, it's essential to define its structure based on the nature of your project. Consider the following elements:

\begin{itemize}
    \item \textbf{Main folders:} Identify the main components or modules of your project and create folders for each.
    \item \textbf{README.md:} Craft a detailed README file that provides an overview of your project, installation instructions, usage guidelines, contribution guidelines, and license information.
    \item \textbf{Documentation:} Dedicate a folder for comprehensive project documentation, including user guides, API references, and developer documentation.
\end{itemize}

\subsection*{Establish Branching Strategy}
Adopting a consistent branching strategy streamlines collaboration and version control. Consider using a branching model such as Gitflow, GitHub Flow, or a customized approach based on your project requirements. Key components of a branching strategy include:

\begin{itemize}
    \item \textbf{Master branch:} Stable production-ready code.
    \item \textbf{Development branch:} Integration branch for ongoing development.
    \item \textbf{Feature branches:} Short-lived branches for implementing new features.
    \item \textbf{Release branches:} Prepare for production releases and hotfixes.
\end{itemize}

\textbf{Example structure:}

\begin{verbatim}
- MyProject
  - docs
    - user_guide.md
    - api_reference.md
    - developer_guide.md
  - src
    - main
      - main.py
      - utils.py
    - tests
      - test_main.py
      - test_utils.py
  - README.md
  - .gitignore
  - LICENSE
  - requirements.txt
  - .github
    - workflows
      - ci.yaml
\end{verbatim}

\subsection*{Utilize Gitignore}
Configure a \texttt{.gitignore} file to exclude unnecessary files and directories from version control. This prevents cluttering the repository with build artifacts, temporary files, IDE-specific configurations, and sensitive information. You can generate a \texttt{.gitignore} file tailored to your project's programming languages and frameworks using \href{https://www.gitignore.io/}{gitignore.io} or create one manually.

\begin{verbatim}
# Example .gitignore file for a Python project
*.pyc
__pycache__
venv/
\end{verbatim}

\subsection*{Categorize Issues and Pull Requests}
Leverage GitHub's issue tracker to manage tasks, bugs, and feature requests effectively. Encourage contributors to categorize and label issues based on priority, type, and status. Similarly, categorize pull requests (PRs) by linking them to corresponding issues and labeling them with appropriate tags (e.g., "bugfix," "enhancement," "documentation").

\subsection*{Implement Continuous Integration/Continuous Deployment (CI/CD)}
Integrate CI/CD pipelines into your repository to automate testing, building, and deployment processes. Services like GitHub Actions, Travis CI, or Jenkins allow you to define workflows triggered by events such as code pushes or PRs. Ensure that your CI/CD configuration is stored in version control and documented for transparency and reproducibility.

\textbf{Example GitHub Actions workflow for Python project:}
\begin{verbatim}
name: CI
on: [push]
jobs:
  build:
    runs-on: ubuntu-latest

    steps:
    - name: Set up Python
      uses: actions/setup-python@v2
      with:
        python-version: '3.x'

    - name: Install dependencies
      run: pip install -r requirements.txt

    - name: Run tests
      run: pytest
\end{verbatim}

\subsection*{Embrace Code Reviews}
Foster a culture of code review by requiring all changes to undergo peer review before merging. GitHub's pull request feature facilitates collaborative code reviews, enabling reviewers to provide feedback, suggest improvements, and approve changes. Encourage reviewers to focus on code quality, readability, maintainability, and adherence to coding standards.
\end{document}